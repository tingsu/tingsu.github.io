%%%%%%%%%%%%%%%%%%%%%%%%%%%%%%%%%%%%%%%%%%%%%%%%%%%%%%%%%%%%%%%%%%%%%%%%%%%%%%%%
% Medium Length Graduate Curriculum Vitae
% LaTeX Template
% Version 1.2 (3/28/15)
%
% This template has been downloaded from:
% http://www.LaTeXTemplates.com
%
% Original author:
% Rensselaer Polytechnic Institute 
% (http://www.rpi.edu/dept/arc/training/latex/resumes/)
%
% Modified by:
% Daniel L Marks <xleafr@gmail.com> 3/28/2015
%
% Important note:
% This template requires the res.cls file to be in the same directory as the
% .tex file. The res.cls file provides the resume style used for structuring the
% document.
%
%%%%%%%%%%%%%%%%%%%%%%%%%%%%%%%%%%%%%%%%%%%%%%%%%%%%%%%%%%%%%%%%%%%%%%%%%%%%%%%%

%-------------------------------------------------------------------------------
%	PACKAGES AND OTHER DOCUMENT CONFIGURATIONS
%-------------------------------------------------------------------------------

%%%%%%%%%%%%%%%%%%%%%%%%%%%%%%%%%%%%%%%%%%%%%%%%%%%%%%%%%%%%%%%%%%%%%%%%%%%%%%%%
% You can have multiple style options the legal options ones are:
%
%   centered:	the name and address are centered at the top of the page 
%				(default)
%
%   line:		the name is the left with a horizontal line then the address to
%				the right
%
%   overlapped:	the section titles overlap the body text (default)
%
%   margin:		the section titles are to the left of the body text
%		
%   11pt:		use 11 point fonts instead of 10 point fonts
%
%   12pt:		use 12 point fonts instead of 10 point fonts
%
%%%%%%%%%%%%%%%%%%%%%%%%%%%%%%%%%%%%%%%%%%%%%%%%%%%%%%%%%%%%%%%%%%%%%%%%%%%%%%%%

\documentclass[margin]{res}  

% Default font is the helvetica postscript font
\usepackage{helvet}
\usepackage{enumitem}
\usepackage{hyperref}
\usepackage{color}

% Increase text height
\textheight=700pt

\begin{document}

%-------------------------------------------------------------------------------
%	NAME AND ADDRESS SECTION
%-------------------------------------------------------------------------------
\name{Ting Su}

% Note that addresses can be used for other contact information:
% -phone numbers
% -email addresses
% -linked-in profile

\address{\\\textbf{Homepage}: \url{http://tingsu.github.io/}
\\\textbf{Office}: CNB H101.1, Universitätstrasse 6, 8092 Zürich, Switzerland\\
}
\address{\\\textbf{Email}: tsuletgo@gmail.com \\\textbf{Tel}: (+86)13501965682\\}

% Uncomment to add a third address
%\address{Address 3 line 1\\Address 3 line 2\\Address 3 line 3}
%-------------------------------------------------------------------------------

\begin{resume}

%-------------------------------------------------------------------------------
%	EDUCATION SECTION
%-------------------------------------------------------------------------------
\section{RESEARCH\\INTERESTS}
{My research interests mainly involve \emph{software engineering}, \emph{programming language}, \emph{software security} and \emph{artificial intelligence}. I am particularly interested in developing novel program analysis, testing and verification techniques to improve software reliability, security, and development productivity. 

In my research, I devote myself to convert these techniques into \emph{practical tools} (not just research prototypes), and benefit both academia and industry.
My research work has improved the reliability and security of different software systems, including \emph{mobile applications}, \emph{embedded software}, \emph{compilers and program analyzers}, \emph{deep learning systems}, etc. 
My PhD thesis tackles the problem of how to leverage control- and data-flow criteria to effectively and automatically generate test data for embedded software.}

\section{RESEARCH\\IMPACTS}
{\textbf{Publications}: \#Total Papers: \textbf{29}, \#Top-tier Papers(CCF-A/SCI-T1) : \textbf{16}, Google Scholar Citation: 387
	
 \textbf{Awards}: \textbf{3} ACM SIGSOFT Distinguished Paper Awards (ICSE 2018, ASE 2018, ASE 2019);
 \textbf{2} Best Research Tool Awards (NASAC 2017, NASAC 2018), and \textbf{1 Golden Medal} of ACM Student Research Competition (ICSE 2016), Nomination for \textbf{Distinguished PhD Thesis} (CCF, 2017)
 
 \textbf{Techniques/Tools}: 
 (1) \textbf{Stoat} (\url{https://github.com/tingsu/stoat}): the \emph{state-of-the-art} automated functional fuzzing tool for mobile apps in the world. It has found \textbf{8000} fatal software errors from \textbf{9000} mobile apps, including \textbf{WeChat}, \textbf{Gmail} and \textbf{Google+}. Prof. Tao Xie (IEEE Fellow), Prof. Andreas Zeller (ACM Fellow), Prof. Jian Lv (Academician of Chinese Academy of Sciences) gave very positive remarks on my work. This work secured \emph{NTUitive GAP Fund with S\$ 240,250} for commercialization. Stoat is widely used by world-leading universities, including UIUC, NUS, UC Davis/Irvine, Peking University, Nanjing University.
 (2) \textbf{SmartRocket Unit} (\url{https://www.ticpsh.com/air.html}): the \emph{world-leading and first} automated intelligent testing tool for embedded industrial software in China. My Ph.D research work was successfully converted into this commercial product. It is now serving more than \textbf{10} domestic industrial companies (including China Academy of Space Technology, CASCO Signal Ltd, Guangzhou Automobile Group), tested more than \textbf{3 millions} lines of code, and recently received the \textbf{German PUV Rheinland security certificate}.
 
 \textbf{Media Reports}:
 1. Stoat: Golden Medal of ACM Student Research Competition, \url{http://www.sei.ecnu.edu.cn/Data/View/2175} (by ECNU)
 2. Stoat: Best Research Tool Award, \url{http://www.sohu.com/a/207736248_610499} (by Software Engineering Research and Practice)
 3. Large-scale fault study for Android apps, \url{https://mp.weixin.qq.com/s/zinnpUWWRVmguaoUoETCvg} (by CodeWisdom)
 4. Automatic GUI Implementation, \url{https://www.jiqizhixin.com/articles/2018-12-02-2} (by jiqizhixin)
}

\section{EMPLOYMENT}
\textbf{ETH Zurich, Switzerland \hfill{June 2019 - Now}\\}
\normalfont{Postdoctoral Scholar\\Advisor: Prof. Zhendong Su}


\textbf{Nanyang Technological University, Singapore \hfill{Oct 2016 - May 2019}\\}
\normalfont{Postdoctoral Research Fellow\\Advisor: Prof. Yang Liu}

\textbf{Rolls-Royce Research Lab, Singapore \hfill{Dec 2016 - Oct 2017}\\}
\normalfont{Research Scientist}

\textbf{Synopsys Shanghai Co. Ltd, China \hfill{July 2015 - Sep 2015}\\}
\normalfont{Software Engineer}

\section{EDUCATION}
\textbf{East China Normal University, China \hfill{Sep 2011 - July 2016}\\}
\normalfont{Ph.D. in Computer Science\\
Dissertation: \emph{Automated Coverage Criteria-based Test Data Generation: Approaches and Implementations}\\
Advisor: Prof. Jifeng He and Prof. Geguang Pu}

\textbf{University of California, Davis, USA \hfill{Feb 2014 - Feb 2015}\\}
\normalfont{Visisting Ph.D. student\\Advisor: Prof. Zhendong Su}

\textbf{East China Normal University, China \hfill{Sep 2007 - July 2011}\\}
\normalfont{B.S. in Software Engineering}

\section{HONORS AND\\AWARDS}
{
\normalfont{ACM SIGSOFT Distinguished Paper Award (ASE 2019, CCF-A), ACM, 2019}\\
({\small{\emph{one of 6 selected papers from total 409 submissions}}})\\
\normalfont{ACM SIGSOFT Distinguished Paper Award (ICSE 2018, CCF-A), ACM, 2018}\\
({\small{\emph{one of 8 selected papers from total 502 submissions}}})\\
\normalfont{ACM SIGSOFT Distinguished Paper Award (ASE 2018, CCF-A), ACM, 2018}\\
({\small{\emph{one of 6 selected papers from total 346 submissions}}})\\
\normalfont{Nomination for Distinguished PhD Thesis, China Computer Federation (CCF), 2017}\\
({\small{\emph{only 20 candidates were selected as finalist nationwide}}})\\
\normalfont{First Place of Research Prototype Tool Competition (NASAC 2017), CCF, 2017}\\
({\small{\emph{NASAC is the flagship annual software engineering and system conference in China}}})\\
\normalfont{Third Place of Research Prototype Tool Competition (NASAC 2018), CCF, 2018}\\
\normalfont{Golden Medal of ACM Student Research Competition (ICSE 2016, CCF-A), ACM, 2016}\\
\normalfont{ACM SIGSOFT CAPS Award (ICSE 2015), ACM, 2015}\\
\normalfont{National Scholarship for Doctorate, China, 2014}
}

\section{RESEARCH\\GRANTS}
\begin{itemize}[leftmargin=*]
\item \normalfont{NTUitive, NTUitive GAP Fund, ``Cloud-based Mobile App Testing Service'', Jan
2018-June 2019, NGF-2017-030033, S\$ 240,250. \textbf{Co-PI} (with PI: Prof. Yang Liu from NTU, Singapore). \textbf{Finished}}
\item \normalfont{NTU SCSE, Research Demo Fund, ``Distributed and Automated Mobile App Testing Platform'', Jan 2018-June 2018, S\$ 50,000. \textbf{Co-PI} (with PI: Prof. Yang Liu, and Co-PI: Guozhu Meng from NTU, Singapore). \textbf{Finished}}
\item \normalfont{Continental, Continental Innovation Project, ``Investigation of AI Techniques for Software Engineering'', Jan 2019-May 2019, S\$ 75,000. \textbf{Co-PI} (with PI: Prof. Shang-Wei Lin and Co-PI: Yang Liu from NTU, Singapore). \textbf{Finished}}
\end{itemize}

%\textbf{LNM Institute of Information Technology}, Jaipur\\
%{\sl Bachelor of Technology (B. Tech)}, Computer Science and Engineering(CSE)\\
%Expected May, 2020
%\hfill CGPA: 7.34/10.00\\

%\textbf{St. Joseph Public School}, Kota (Raj.) \\
%{\sl Class XII (Senior Secondary Examination)}, CBSE\\
%July 2015
%\hfill Aggregate 87.8\%

%\textbf{St. Joseph Public School}, Kota (Raj.) \\
%{\sl Class X (Secondary Examination)}, CBSE\\
%July 2013
%\hfill CGPA: 10/10

\section{PROJECTS}
\normalfont{\textbf{Mobile App Analysis, Testing and Security\hfill{2015 - now}}}\\
I lead several research work on analyzing, testing, debugging, and securing mobile applications. The \emph{key goal} is to sharpen the competitive edge of mobile apps, and ensure their reliability and security.
In particular, I spent two years and developed a state-of-the-art GUI testing tool \emph{\textbf{Stoat}} (\url{https://github.com/tingsu/stoat}) for Android apps.
Stoat has been deployed to extensively test \underline{9,000+} open-source and commercial apps and found \underline{8,000+} fatal crashes in one year. Stoat has contributed to such popular apps as \texttt{WeChat}, \texttt{Google+}, \texttt{Gmail} that have billions of users. Stoat won the \emph{Golden Medal of ACM Student Research Competition} in ICSE 2016, \emph{First Place of Research Tool Competition} in NASAC 2017, and received \emph{NTUitive Gap Fund Grant} S\$240,250 in 2018 for commercialization potential. Stoat was recognized as ``\emph{a state-of-the-art tool, (which) is able to trigger the highest number of unique crashes (in Android apps)}'' by Prof. Tao Xie (IEEE Fellow) in their ASE'18 paper; ``\emph{Stoat is effective for discovering unique
crashes}`` by Prof. Andreas Zeller (ACM Fellow) in their ISSTA'19 paper.
Stoat has also been widely used by the researchers from many renowned institutes worldwide, e.g., UIUC, UC Irvine, NUS, ETH Zurich, Peking University, Nanjing University. See the demo of Stoat testing platform: \url{https://youtu.be/41jzFM7WhP4}

I also lead the first, largest and most comprehensive fault study for Android apps (\url{https://crashanalysis.github.io/}), based on the bugs detected by Stoat and the data from open-source hosting platforms (e.g., \texttt{Github}, \texttt{Google Code}). This study involves \underline{2,486} open-source apps, \underline{3,230} commercial apps, and \underline{135} professional app developers worldwide. We investigate the key question ``\emph{why an Android app may crash?}'', understand how app developers analyze, test, reproduce and fix app crashes, and motivate several follow-up research directions. This work won an \emph{ACM SIGSOFT Distinguished Paper Award} in ICSE 2018 with \textbf{three strong acceptance}. Our dataset and benchmark have been requested by more than 13 institutes worldwide (e.g., CMU, UNSW, Monash, HKUST). Following this study, we designed more effective testing techniques (Stoat+ and APEChecker) for detecting intricate bug types (e.g., asynchronous programming errors).

We also conducted a large-scale automated security risk assessment for mobile banking apps. This study revealed \underline{2,157} security weaknesses in \underline{693} banking apps worldwide. To date, \underline{21} banking entities have confirmed \underline{126} weaknesses, and \underline{52} weaknesses have been patched. This work has been well-acknowledged by some bank entities from HonKong, Singapore and India, e.g., \texttt{HSBC}, \texttt{OCBC}, \texttt{DBS} and \texttt{BHIM}. The tool \textbf{Ausera} won \emph{Third Place of Research Tool Competition} in NASAC 2018, and now has been integrated into one security scanning product of \href{https://scantist.com/}{\texttt{Scantist}}. 

\normalfont{\textbf{Automatic Test Data Generation for Embedded Software \hfill{2014 - now}}}\\
Since my Ph.D., I continuously lead the research work on leveraging \emph{dynamic symbolic execution} techniques to automatically generate test data for industrial embedded software. The \emph{key goal} is to reduce human testing efforts and detect functional bugs as early as possible (especially at the unit testing stage).
I developed and maintained an academic research tool \textbf{CAUT} that automatically generates test data to satisfy different testing adequacy criteria. Later, as a co-founder, I significantly contributed to \textbf{SmartRocket Unit} (\url{https://www.ticpsh.com/air.html}), a commercial automated unit testing tool inspired by CAUT. SmartUnit fully supports unit testing for statement, branch, boundary value and MC/DC coverage, and has been applied in several industry-scale embedded control systems, and accumulatively tested \underline{3 millions lines} of C code.  
It is now serving \underline{more than ten} Chinese industrial companies, including \texttt{CASCO} (the best railway signal
corporation in China), \texttt{GAC MOTOR} (one of biggest car manufacturers in China), \texttt{CAST} (the main spacecraft production agency in China). It recently received the German PUV Rheinland security certificate.

\normalfont{\textbf{Testing Compilers and Program Analyzers \hfill{2015 - now}}}\\
I also worked on testing compilers (e.g., Java Virtual Machines) and program analyzers (e.g., software model checkers, SAT/SMT solvers). The \emph{key goal} is to ensure the correctness and reliability of many fundamental development tools we very depend on.
We proposed two stress-testing techniques \emph{\textbf{classming}} and \emph{\textbf{classfuzz}} for JVMs. We discovered \underline{62+} JVM discrepancies in the start-up process of \texttt{Oracle}'s HotSpot and \texttt{IBM}'s J9, which lead to  several clarifications and changes to the Java SE 8 edition of the JVM specification; We revealed \underline{30+} fatal bugs in \texttt{Oracle}'s HotSpot and \texttt{IBM}'s J9. In particular, we found one highly-critical security vulnerability (CVE-2017-1376, CVSS base score \underline{9.8}), which directly affects \underline{6} IBM's products and \underline{99} derivative products.

To validate the correctness of software model checkers (the implementations of a well-known automatic software verification technique proposed by Turing Award Winner E. M. Clarke et al.), we proposed an effective testing technique \textbf{MCFuzz} (\url{https://github.com/MCFuzzer/MCFuzz}), which is now \underline{the only available, effective approach} in the world to test software model checkers. It has extensively tested three state-of-the-art and state-of-the-practice C software model checkers, \texttt{CPAchecker}, \texttt{CBMC}, and \texttt{SeaHorn}. We have found bugs in all
three model checkers. To date, we have reported a total of \underline{62} unique bugs --- \underline{59} were previously unknown, \underline{58} have been confirmed, and \underline{20} have been fixed. We are well-appreciated by the model checker developers and our tests have been integrated as their standard benchmarks.

\normalfont{\textbf{Business Process Model Mining and Validation \hfill{2017 - 2018}}}\\
At the Rolls-Royce research lab, I worked on developing an automated business process model translation and mining tool chain to validate the correctness of in-house business process, especially for the airplane engine development process.  

\normalfont{\textbf{Setting Exploration Automation \hfill{2015}}}\\
At Synopsys, I worked on developing a distributed testing framework to efficiently generate optimal parameter configurations for Synopsys's in-house hardware verification tools. This project was recognized as \emph{an innovation project} in Synopsys.

\section{PUBLICATIONS}
\normalfont{\textbf{\underline{Summary}}}
\begin{itemize}[leftmargin=*]
    \item Note that the tradition of publication in computer science, and my field in particular, is mainly
to publish full papers in peer-reviewed conference proceedings. ICSE, FSE, ASE, PLDI, and OOPSLA are fiercely competitive and widely accepted as the top-tier conferences in my research area. Top-tier conferences have a typical acceptance rate around 20\%. TSE, TOSEM, CSUR are top-tier journals in my research area.
    \item In each publication, the label ``\textbf{*}'' denotes I am the corresponding and the main author, ``$^{\textbf{\#}}$'' denotes I am the co-first author and make equal contribution with the first author,  ``$^{\textbf{s}}$'' denotes the students I directly supervised when the work was conducted. \emph{\textbf{\underline{Without any specific annotations, each paper is full research paper}}}.
    \item Basically, for all papers, I have made significant contributions to the idea, research, discussions and paper writing. In particular, I am the corresponding and the main author of the following \textbf{11} \emph{top-tier} published papers: $[$\textbf{FSE-19}$]$, $[$\textbf{ICSE-19a}$]$, $[$\textbf{ICSE-18a}$]$, $[$\textbf{ICSE-18b}$]$, $[$\textbf{ICSE-18c}$]$, $[$\textbf{ASE-18}$]$, $[$\textbf{FSE-18}$]$, $[$\textbf{FSE-17}$]$, $[$\textbf{CSUR-17}$]$, $[$\textbf{ICSE-16}$]$, $[$\textbf{ICSE-15}$]$.
\end{itemize}

\normalfont{\textbf{\underline{Public Profiles}}}
\begin{itemize}[leftmargin=*]
    \item Google Scholar: \url{https://scholar.google.com/citations?user=GHPWI7oAAAAJ}
    \item DBLP: \url{https://dblp.uni-trier.de/pers/hd/s/Su:Ting}
    \item Personal Page: \url{http://tingsu.github.io/}
\end{itemize}

\normalfont{\textbf{\underline{Highlight of Published Papers}}:} Total number of published papers (29), among them: ICSE(8), FSE(3), ASE(3), CSUR(1), PLDI(1)

\normalfont{\textbf{\underline{Peer-reviewed Journal Papers}}}
\begin{enumerate}[leftmargin=*]
	\item $[$\textbf{TSE-20}$]$ \textbf{Ting Su}, Lingling Fan, Sen Chen, Yang Liu, Lihua Xu, Geguang Pu, and Zhendong Su. Why My App Crashes? Understanding and Benchmarking Framework-specific Exceptions of Android apps. IEEE Transactions on Software Engineering. TSE 2020. (IF: 3.33, CCF-A), \emph{Minor Revision}.
    \item $[$\textbf{CSUR-17}$]$ \textbf{Ting Su}, Ke Wu, Weikai Miao, Geguang Pu, Jifeng He, Yuting Chen, Zhendong Su. A Survey on Data-Flow Testing. ACM Computing Surveys, 2017 (IF: 6.748. SCI Rank 1)
    \item $[$\textbf{SCIS-16}$]$ \textbf{Ting Su}, Geguang Pu, Weikai Miao, Jifeng He, Zhendong Su. Automated Coverage-driven Testing: Combining Symbolic Execution and Model Checking. SCIENCE CHINA Information Sciences, 2016 (IF: 1.628, CCF-B) (\emph{Invited Special Session Paper})
    \item $[$\textbf{IJCM-13}$]$ Yongxin Zhao, Hao Xiao, Zheng Wang, Geguang Pu, \textbf{Ting Su}. The Semantics and Verification of Timed Service Choreography. International Journal of Computer Mathematics. IJCM 2013
\end{enumerate}

\normalfont{\textbf{\underline{Peer-reviewed Conference Papers}}}
\begin{enumerate}[leftmargin=*]
	\item $[$\textbf{FSE-17}$]$ \textbf{Ting Su}, Guozhu Meng, Yuting Chen, Ke Wu, Weiming Yang, Yao Yao, Geguang Pu, Yang Liu, Zhendong Su. Guided, Stochastic Model-Based GUI Testing of Android Apps. The 11th joint meeting of the European Software Engineering Conference and the ACM SIGSOFT Symposium on the Foundations of Software Engineering. FSE 2017. (CCF-A) 
	
	\textbf{\emph{\textcolor{red}{First Place of Research Prototype Tool Competition in NASAC 2017}}}.
	\item $[$\textbf{ICSE-16}$]$ \textbf{Ting Su}. FSMdroid: Guided GUI Testing of Android Apps. The 38th International Conference on Software Engineering. ICSE 2016. (CCF-A, Student Research Competition Paper) 
	
	\textbf{\emph{\textcolor{red}{Golden Medal of ACM Student Research Competition}}}.
	\item $[$\textbf{ICSE-15}$]$ \textbf{Ting Su}, Zhoulai Fu, Geguang Pu, Jifeng He, Zhendong Su. Combining Symbolic Execution and Model Checking for Data Flow Testing. 37th {IEEE/ACM} International Conference on Software Engineering. ICSE 2015. (CCF-A)
	\item $[$\textbf{QRS-14}$]$ \textbf{Ting Su}, Siyuan Jiang, Geguang Pu, Bin Fang, Jifeng He, Jun Yan, Jianjun Zhao. Automated Coverage-Driven Test Data Generation Using Dynamic Symbolic Execution. Eighth International Conference on Software Security and Reliability. SERE 2014. (now renamed as QRS)
    \item $[$\textbf{FSE-19}$]$ Chengyu Zhang$^{\textbf{s\#}}$, \textbf{Ting Su$^{\textbf{\#*}}$}, Yichen Yan, Fuyuan Zhang, Geguang Pu and Zhendong Su. Finding and Understanding Bugs in Software Model Checkers. The 27th ACM Joint European Software Engineering Conference and Symposium on the Foundations of Software Engineering. FSE 2019. (CCF-A)
    \item $[$\textbf{ICSE-18a}$]$ Lingling Fan$^{\textbf{s\#}}$, \textbf{Ting Su$^{\textbf{\#*}}$}, Sen Chen, Guozhu Meng, Yang Liu, Lihua Xu, Geguang Pu and Zhendong Su. Large-Scale Analysis of Framework-Specific Exceptions in Android Apps. The 40th International Conference on Software Engineering. ICSE 2018. (CCF-A) 
    
    \textbf{\emph{\textcolor{red}{ACM SIGSOFT Distinguished Paper Award}}}
    \item $[$\textbf{ICSE-19a}$]$ Yuting Chen, \textbf{Ting Su$^{\textbf{*}}$}, Zhendong Su. Deep Differential Testing of JVM Implementations. The 41st ACM/IEEE International Conference on Software Engineering. ICSE 2019. (CCF-A)
    \item $[$\textbf{ICSE-18b}$]$ Chunyang Chen$^{\textbf{s}}$, \textbf{Ting Su$^{\textbf{*}}$}, Guozhu Meng, Zhenchang Xing and Yang Liu. From UI Design Image to GUI Skeleton: A Neural Machine Translator to Bootstrap Mobile GUI Implementation. The 40th International Conference on Software Engineering. ICSE 2018. (CCF-A)
    \item $[$\textbf{ASE-18a}$]$ Lingling Fan$^{\textbf{s}}$, \textbf{Ting Su$^{\textbf{*}}$}, Sen Chen, Guozhu Meng, Yang Liu, Lihua Xu, Geguang Pu. Efficiently Manifesting Asynchronous Programming Errors in Android Apps. The 33rd IEEE/ACM International Conference on Automated Software Engineering. ASE 2018. (CCF-A)
    \item $[$\textbf{FSE-18}$]$ Sen Chen$^{\textbf{s}}$, \textbf{Ting Su$^{\textbf{*}}$}, Lingling Fan, Guozhu Meng, Minhui Xue, Yang Liu, Lihua Xu. Are Mobile Banking Apps Secure? What Can be Improved? The 26th joint meeting of the European Software Engineering Conference and the ACM SIGSOFT Symposium on the Foundations of Software Engineering. FSE 2018. (CCF-A) (Full Industry Paper)
    \item $[$\textbf{PLDI-16}$]$ Yuting Chen, \textbf{Ting Su}, Chengnian Sun, Zhendong Su, Jianjun Zhao. Coverage-Directed Differential Testing of JVM Implementations. ACM SIGPLAN Conference on Programming Language Design and Implementation. PLDI 2016. (CCF-A)
    \item $[$\textbf{ICSE-18c}$]$ Chengyu Zhang$^{\textbf{s}}$, Yichen Yan$^{\textbf{s}}$, Hanru Zhou, Yinbo Yao, Ke Wu, \textbf{Ting Su$^{\textbf{*}}$}, Weikai Miao and Geguang Pu. SmartUnit: Empirical Evaluations for Automated Unit Testing of Embedded Software in Industry. The 40th International Conference on Software Engineering. ICSE 2018. (CCF-A) (Full Industry Paper)
    \item $[$\textbf{ASE-19a}$]$ Yan Zheng, Xiaofei Xie, \textbf{Ting Su}, Lei Ma, Jianye Hao, Zhaopeng Meng, Yang Liu, Ruimin Shen, Yinfeng Chen, and Changjie Fan. Wuji: Automatic Online Combat Game Testing Using Evolutionary Deep Reinforcement Learning. The 34th IEEE/ACM International Conference on Automated Software Engineering. ASE 2019. (CCF-A).
    
    \textbf{\emph{\textcolor{red}{ACM SIGSOFT Distinguished Paper Award}}}
    \item $[$\textbf{ICSE-19b}$]$ Yu Zhao, Tingting Yu, \textbf{Ting Su}, Yang Liu, Wei Zheng, Jingzhi Zhang, William G.J. Halfond. ReCDroid: Automatically Reproducing Android Application Crashes from Bug Reports. The 41st ACM/IEEE International Conference on Software Engineering. ICSE 2019. (CCF-A)
    \item $[$\textbf{ICSE-19c}$]$ Sen Chen, Lingling Fan, Chunyang Chen, \textbf{Ting Su}, Wenhe Li, Yang Liu, Lihua Xu. StoryDroid: Automated Generation of Storyboard for Android Apps. The 41st ACM/IEEE International Conference on Software Engineering. ICSE 2019. (CCF-A)
    \item $[$\textbf{TASE-19}$]$ Li Hao, Jianqi Shi, \textbf{Ting Su} and Yanhong Huang. Automated Test Generation for IEC 61131-3 ST Programs via Dynamic Symbolic Execution. The 13th International Symposium on Theoretical Aspects of Software Engineering. TASE 2019.
    \item $[$\textbf{ASE-18b}$]$ Lei Ma, Felix Juefei-Xu, Fuyuan Zhang, Jiyuan Sun, Minhui Xue, Bo Li, Chunyang Chen, \textbf{Ting Su}, Li Li, Yang Liu, Jianjun Zhao, and Yadong Wang. DeepGauge: Multi-Granularity Testing Criteria for Deep Learning Systems. The 33rd IEEE/ACM International Conference on Automated Software Engineering. ASE 2018. (CCF-A) 
    
    \textbf{\emph{\textcolor{red}{ACM SIGSOFT Distinguished Paper Award}}}
    \item $[$\textbf{ICFEM-16}$]$ Weikai Miao, Geguang Pu, Yinbo Yao, \textbf{Ting Su}, Danzhu Bao, Yang Liu, Shuohao Chen and Kunpeng Xiong. Automated Requirements Validation for ATP Software via Specification Review and Testing. The 18th International Conference on Formal Engineering Methods. ICFEM 2016. 
    \item $[$\textbf{KSEM-15}$]$ Ke Wu, Shiping Tang, Geguang Pu, Min Wu, \textbf{Ting Su}. Fm-QCA: A Novel Approach to Multi-value Qualitative Comparative Analysis. 8th International Conference on Knowledge Science, Engineering and Management. KSEM 2015. 
    \item $[$\textbf{APSEC-14}$]$ Yan Shen, Jianwen Li, Zheng Wang, \textbf{Ting Su}, Bin Fang, Geguang Pu and Wangwei Liu. Runtime Verification by Convergent Formula Progression. 21st Asia-Pacific Software Engineering Conference. APSEC 2014. 
    \item $[$\textbf{TASE-13}$]$ Yunhui Peng, Yanhong Huang, Ting Su, Jian Guo. Modeling and Verification of AUTOSAR OS and EMS Application. Seventh International Symposium on Theoretical Aspects of Software Engineering. TASE 2013. 
\end{enumerate}

\normalfont{\textbf{\underline{Papers Under Review and Submission}}}
\begin{enumerate}[leftmargin=*]
	\item $[$\textbf{PLDI-20}$]$ \textbf{Ting Su}, Yichen Yan, Junxin Li, Zhendong Su et al. Property-based functional fuzzing for Android apps. 2020 (CCF-A)
    \item $[$\textbf{TOSEM-19}$]$ \textbf{Ting Su}, Chengyu Zhang, Yichen Yan, Lingling Fan, Geguang Pu, Yang Liu, Zhoulai Fu and Zhendong Su. Towards Efficient Data-flow Test Data Generation. ACM Transactions on Software Engineering and Methodology. TOSEM 2019. (IF: 2.516, CCF-A) \emph{Major Revision}.
    \item $[$\textbf{TDSC-19}$]$ Sen Chen, Guozhu Meng, \textbf{Ting Su}, Lingling Fan, Minhui Xue, Yinxing Xue, Yang Liu, and Lihua Xu. AUSERA: Large-Scale Automated Security Risk Assessment of Global Mobile Banking Apps. IEEE Transactions on Dependable and Secure Computing. 2019. (IF: 4.41, CCF-A), \emph{Major Revision}.
\end{enumerate}

\normalfont{\textbf{\underline{Posters and Workshop Papers}}}
\begin{enumerate}[leftmargin=*]
    \item $[$\textbf{AI4Mobile-19}$]$ Sen Chen, Lingling Fan, \textbf{Ting Su}, Lei Ma, Yang Liu, and Lihua Xu. Automated Cross-Platform GUI Code Generation for Mobile Apps, 1st International Workshop on Artificial Intelligence for Mobile. AI4Mobile 2019.
    \item $[$\textbf{ICSTW-19}$]$ Ruitao Feng, Guozhu Meng, Xiaofei Xie, \textbf{Ting Su}, Yang Liu, Shang-Wei Lin. Learning Performance Optimization from Code Changes for Android Apps. 2019 IEEE International Conference on Software Testing, Verification and Validation Workshops. ICSTW 2019.
    \item $[$\textbf{KLEE-18}$]$ Chengyu Zhang, \textbf{Ting Su}, Yichen Yan, Ke Wu, and Geguang Pu. Towards Efficient Data-flow Test Data Generation Using KLEE. The 1st International KLEE Workshop on Symbolic Execution. KLEE 2018.
\end{enumerate}

\section{STUDENT\\SUPERVISION}
During my postdoc, I have co-supervised several Ph.D/master/undergraduate students at Nanyang Technological University (NTU), ETH Zurich (ETH), and East China Normal University (ECNU). For each of their work, I have made significant contributions to the idea, research, discussions and paper writing.

\normalfont{\textbf{\underline{Ph.D students}}}
\begin{enumerate}[leftmargin=*]
    \item \textbf{Lingling Fan}. Visiting Ph.D student at NTU from ECNU. \\Topic: Bug analysis of mobile apps. \\Published Papers: ICSE'18, ASE'18. \\Co-supervision, supervision percentage: 90\%
    \item \textbf{Chengyu Zhang}. Visiting Ph.D student at ETH from ECNU. \\Topic: Testing software model checkers. \\Published Papers: FSE'19. \\Co-supervision, supervision percentage: 90\%
    \item \textbf{Sen Chen}. Visiting Ph.D student at NTU from ECNU. \\Topic: Security analysis of mobile banking apps. \\Published Papers: FSE'18. \\Co-supervision, supervision percentage: 50\%
    \item \textbf{Jingling Sun}. Ph.D student at ECNU. \\Topic: Configuration analysis of mobile apps. \\Co-supervision, supervision percentage: 90\%
\end{enumerate}
\normalfont{\textbf{\underline{Master and Undergraduate students}}}
\begin{enumerate}[leftmargin=*]
    \item \textbf{YiChen Yan}. Visiting master student at ETH from ECNU. \\Topic: Testing C++ standard library. \\Co-supervision, supervision percentage: 90\%
    \item \textbf{Junxin Li}. master student at ECNU. \\Topic: Semantic testing of mobile apps \\Co-supervision, supervision percentage: 90\%
    \item \textbf{Wee Soon Lee Aaron}. Undergraduate student at NTU. \\Topic: Fuzzing of mobile apps.  \\Co-supervision, supervision percentage: 90\%
\end{enumerate}

\section{TEACHING\\EXPERIENCE}
\begin{enumerate}[leftmargin=*]
	\item Teaching Assistant, Master Seminar, Research Topics in Software Engineering, Fall 2019, ETH Zurich (Prof. Zhendong Su)
    \item Teaching Assistant, CZ2006 Software Engineering, 2017/2018 Semester 1, NTU (Prof. Yang Liu)
    \item Teaching Assistant, Tools of Program Analysis, Testing, and Verification, 2015 Semester 1, ECNU (Prof. Geguang Pu)
\end{enumerate}


\section{PATENTS}
\normalfont{\textbf{\underline{Chinese Patents}}}\\
All these patents are based on my research work.
\begin{enumerate}[leftmargin=*]
    \item Geguang Pu (Ph.D supervisor), \textbf{Ting Su}, Chengyu Zhang, Ke Wu, Yibo Yao, Hanru Zhou. An approach to cut point guided data-flow testing and its system. CN107656863A. Filed November 3, 2016
    \item Geguang Pu (Ph.D supervisor), \textbf{Ting Su}, Ke Wu, Weiming Yang, Qiming Yao, Yao Yao. An stochastic model based testing approach for mobile applications. CN107656864A. Filed November 9, 2016
\end{enumerate}

\section{INVITED TALKS}
\begin{itemize}[leftmargin=*]
   
    \item \textbf{Ting Su}. Large-Scale Analysis of Framework-Specific Exceptions in Android Apps. 12th Innovations in Software Engineering Conference (ISEC 2019, \emph{the flagship software engineering conference in India}), February 2019, Pune, India. (Invited by Indian Chapter of SIGSOFT)
    \item \textbf{Ting Su}. Pushing the Limits of Automated Testing for Mobile Applications. The First International Workshop on Advances in Mobile App Analysis, A-MOBILE 2018 (co-located with ASE 2018), September 2018, Montpellier, France.
    \item \textbf{Ting Su}. Automated GUI Testing of Mobile Applications: Experience and Lessons Learned. ShanghaiTech University, May 2019, Shanghai, China.
    \item \textbf{Ting Su}. Guided, Stochastic Model-based GUI Testing of Android Apps. Cybertech Asia 2018. March 2018, Singapore.
    \item \textbf{Ting Su}. Stoat: A cloud-based Mobile App Testing Service. Defence Science and Technology Agency, Singapore (DSTA) March 2018, Singapore (Invited by DSTA)
\end{itemize}

\section{PROFESSIONAL\\ACTIVITIES}
\normalfont{\textbf{\underline{Program Committee Member}}}
\begin{itemize}[leftmargin=*]
	\item The 42nd International Conference on Software Engineering (ICSE), SEIP, 2020. (CCF-A)
    \item IEEE International Conference on Software Testing, Verification and Validation (ICST), Industry Track, 2019-2020. (CCF-C)
    \item International Symposium on Theoretical Aspects of Software Engineering (TASE), 2019-2020
    \item National Software Application Conference (NASAC), China, 2019. (CCF-C)
    \item International Workshop on Advances in Mobile App Analysis (A-Mobile), 2019-2020
    \item International Workshop on User Interface Test Automation and Testing Techniques for Event Based Software (INTUITESTBEDS), 2019
    \item International Workshop on Testing: Academia-Industry Collaboration, Practice and Research Techniques (TAIC-PART), 2019-2020
\end{itemize}
\normalfont{\textbf{\underline{Invited Journal Reviewer}}}
\begin{itemize}[leftmargin=*]
	\item Automated Software Engineering Journal (ASE), CCF-B
    \item Information and Software Technology (IST), CCF-B
    \item Software Quality Journal (SQJ), CCF-B
    \item Journal of Software: Evolution and Process, CCF-B
    \item IEEE Access
\end{itemize}
\normalfont{\textbf{\underline{External Reviewer (selected)}}}
\begin{itemize}[leftmargin=*]
    \item Joint meeting of the European Software Engineering Conference and the ACMSIGSOFT Symposium on the Foundations of Software Engineering (FSE), 2019
    \item International Conference on Tools and Algorithms for the Construction and Analysis of Systems (TACAS) 2018, 2019
    \item International Conference onSoftware Engineering (ICSE), 2019
    \item IEEE/ACM  International  Conference  on  Automated  Software  Engineering (ASE), 2018
    \item ACM SIGSOFT International Symposium on Software Testing and Analysis (ISSTA), 2018
    \item IEEE International Conference on Software Analysis, Evolution and Reengineering (SANER), 2019
    \item IEEE Transaction on Reliability, Frontiers of Computer Science, Journal of Computer Science and Technology (JCST)
\end{itemize}

%-------------------------------------------------------------------------------
%	COMPUTER SKILLS SECTION
%-------------------------------------------------------------------------------
\section{LANGUAGES}
Proficient in English and Mandrin.

%-------------------------------------------------------------------------------
% Modify the format of each position
\begin{format}
\title{l}\\
\dates{l}\location{r}\\
\body\\
\end{format}
%-------------------------------------------------------------------------------





%-------------------------------------------------------------------------------
%\section{REFERENCES}
%
%\normalfont{Dr. Zhendong Su}\\
%\emph{Professor}, Department of Computer Science, ETH Zurich\\
%Universitätstrasse 6, 8092 Zürich, Switzerland\\
%Email: zhendong.su@inf.ethz.ch\\
%\\
%Dr. Yang Liu\\
%\emph{Associate Professor}, School of Computer Science and Engineering\\
%Nanyang Technological University, Singapore\\
%50 Nanyang Avenue, Singapore 639798\\
%Email: yangliu@ntu.edu.sg \\
%\\
%Dr. Geguang Pu \\
%\emph{Professor}, School of Computer Science and Software Engineering\\
%East China Normal University\\
%3663 Zhongshan Road (North) Shanghai, 200062, China \\
%Email: ggpu@sei.ecnu.edu.cn\\






\end{resume}
\(\)\end{document}